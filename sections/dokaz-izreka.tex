\documentclass[../TG_magistrsko_delo_sections.tex]{subfiles}
\graphicspath{{\subfix{../images/}}}

\begin{document}
V tem poglavju bomo dokazali glavni del izreka Šarkovskega. Vemo že, da izrek velja, če obstaja točka cikla, ki ne menja strani. V primeru, da vse točke menjajo strani, bomo podobno kot v primeru~\ref{primer4} cikel razdelili na levo in desno polovico. Vsaka polovica tvori cikel za funkcijo $f^2$. Informacijo o ciklih funkcije $f^2$ bomo nato prenesli na cikle funkcije $f$.

\begin{trditev}\label{trd:realcvtm}
Naj bosta $m$ in $l$ naravni števili v relaciji $m \triangleright l$ in naj bo $\mathcal{O}$ $m$-cikel. Potem obstaja $\mathcal{O}$-vsiljena elementarna $l$-zanka $\mathcal{O}$-intervalov in posledično točka s periodo $l$.
\end{trditev}

\begin{proof}
%Vemo že, da dokaz velja, če cikel $\mathcal{O}$ vsebuje Štefanovo zaporedje. To se zgodi vsakič, ko vsaj ena točka cikla $\mathcal{O}$ menja stran. Trditev moramo dokazati še v primeru, ko vsaka točka cikla $\mathcal{O}$ menja stran. Predpostavimo, da je cikel $\mathcal{O}$ tak, da vsaka točka iz cikla menja stran. 
Izrek bomo dokazali s pomočjo indukcije na število $m$. 

Če je $m=1$, je trditev avtomatično izpolnjena, saj je 1 zadnji člen zaporedja Šarkovskega in edino število $l$, za katerega velja $l \triangleleft 1$ je 1. 

Predpostavimo, da izrek velja za vse cikle, katerih dolžina je krajša od $m$. Radi bi dokazali, da velja tudi za poljuben $m$-cikel $\mathcal{O}$. Če obstaja točka iz cikla $\mathcal{O}$, ki ne menja strani, potem je po trditvi~\ref{trd:tnmsoc} resična tudi trditev~\ref{trd:realcvtm}. V nasprotnem primeru vse točke cikla $\mathcal{O}$ menjajo strani. Označimo najmanjšo točko cikla $\mathcal{O}$  z $L$ in največjo točko cikla $\mathcal{O}$ z $R$. Množica $\mathcal{O}_L$ vsebuje vse točke iz cikla $\mathcal{O}$, ki ležijo levo od centra $c$, množica $\mathcal{O}_R$ pa vsebuje vse točke, ki ležijo desno od centra $c$. Ker vse točke iz cikla $\mathcal{O}$ menjajo strani, funkcija $f$ slika množico $\mathcal{O}_L$ v množico $\mathcal{O}_R$ in obratno. Funkcija $f|_{\mathcal{O}_L}$ je bijekcija iz množice $\mathcal{O}_L$ v množico $\mathcal{O}_R$ in funkcija $f|_{\mathcal{O}_R}$ je bijekcija iz množice $\mathcal{O}_R$ v množico $\mathcal{O}_L$. Ugotovimo, da množici $\mathcal{O}_L$ in $\mathcal{O}_R$ vsebujeta enako število točk, zato je število $m$ sodo in obstaja naravno število $n$, za katerega je $m = 2 n$. Ker je $m$ sodo število, je lahko neko naravno število $l$ v relaciji $l \triangleleft m$ samo, če je $l=1$ ali pa je $l$ sodo število. V drugem primeru obstaja tako naravno število $k$, za katerega je $l = 2 k$. Iz zgornjega razmisleka in iz trditve~\ref{trd:doubling} sledi, da je neko naravno število $l$ v relaciji $l \triangleleft m$ natanko tedaj, ko je $l=1$ ali pa je $l=2k$ in je število $k$ v relaciji $k \triangleleft n$. To pomeni, da moramo pokazati, da ima $f$ elementarno 1-zanko in elementarno $\mathcal{O}$-vsiljeno $2k$-zanko $\mathcal{O}$-intervalov za vsako naravno število $k$ za katerega velja relacija $k \triangleleft n$. Elementarno 1-zanko dobimo s pomočjo intervala $[p, q]$. Točka $p$ je največja točka množice $\mathcal{O}_L$ in točka $q$ je najmanjša točka množice $\mathcal{O}_R$. Ker točka $f(p)$ leži v množici $\mathcal{O}_R$ in točka $f(q)$ leži v množici $\mathcal{O}_L$ dobimo elementarno 1-zanko $[p, q] \to [p, q]$.
Pri dokazovanju obstoja $2k$-zanke za vsako naravno število $k$, ki ustreza relaciji $k \triangleleft n$, si bomo pomagali z indukcijsko predpostavko. Opazimo, da sta množici $\mathcal{O}_L$ in $\mathcal{O}_R$  cikla dolžine $n$ za funkcijo $f^2$. Ker je dolžina obeh ciklov manjša od $m$, lahko uporabimo indukcijsko predpostavko. Če indukcijsko predpostavko uporabimo na ciklu $\mathcal{O}_R$, ugotovimo, da za vsako naravno število $k$, za katerega je $k \triangleleft n$, obstaja elementarna $\mathcal{O}_R$-vsiljena $k$-zanka $\mathcal{O}_R$ intervalov za funkcijo $f^2$. Pokazati moramo, da te zanke zagotavljajo obstoj elementarnih $l$-zank za funkcijo $f$. Poglejmo si poljubno elementarno $k$-zanko $\mathcal{O}_R$ intervalov za funkcijo $f^2$:
\begin{equation}\label{kloop}
I_0 \xrightarrow{f^2} I_1 \xrightarrow{f^2} I_2 \xrightarrow{f^2} \cdots \xrightarrow{f^2} I_{k-1} \xrightarrow{f^2} I_0.
\end{equation}
Za vsako naravno število $0 \leq i < k$ označimo najkrajši zaprti interval, ki vsebuje množico $f(I_i \cap \mathcal{O}) \subseteq \mathcal{O}_L$, z $I_i'$. Intervali $I_i'$ so $\mathcal{O}$-intervali za katere veljajo relacije pokritja $I_i \xrightarrow{f} I_i'$. Če interval $I_0$ označimo z $I_k$ lahko naradimo naslednji razmislek. Za vsako naravno število $0 \leq i < k$ lahko zapišemo $\mathcal{O}_R$-vsiljene relacije pokritja $I_i \xrightarrow{f^2} I_{i+1}$, zato obstajata taki točki $a_i, b_i \in I_i \cap \mathcal{O}_R$, da interval $I_{i+1}$ leži v intervalu omejenim s točkama $f^2(a_i)$ in $f^2(b_i)$. Točki $a_i' := f(a_i)$ in $b_i' := f(b_i)$ ležita v množici $I_i' \cap \mathcal{O}$ in zaprt interval omejen s točkama $f(a_i') = f^2(a_i)$ in $f(b_i') = f^2(b_i)$ vsebuje interval $I_{i+1}$. Dobili smo $\mathcal{O}$ vsiljeno relacijo pokritja $I_i' \xrightarrow{f} I_{i+1}$. S pomočjo zgornjih relacij pokritja lahko zapišemo naslednjo $\mathcal{O}$-vsiljeno $l$-zanko:
\begin{equation}\label{lloop}
I_0 \xrightarrow{f} I'_0 \xrightarrow{f} I_1 \xrightarrow{f} I'_1 \xrightarrow{f} \cdots \xrightarrow{f} I_{k-1} \xrightarrow{f} I'_{k-1} \xrightarrow{f} I_0
\end{equation}
Prepričajmo se, da je zanka~\eqref{lloop} elementarna. Naj bo točka $x$ periodična točka za funkcijo $f$, ki sledi zanki~\eqref{lloop}. Točka $x$ je periodična točka za funkcijo $f^2$, ki sledi zanki~\eqref{kloop}. Torej, točka $x$ ima periodo $k$ za funkcijo $f^2$, kar pomeni, da $k$ točk $f$-orbite leži v množici $\mathcal{O}_R$. Ker intervali v zanki~\eqref{lloop} ležijo izmenično na levi oziroma desni strani točke $c$, tudi iteracije točke $x$ ležijo izmenično na levi oziroma desni strani točke $c$. To pomeni, da $k$ točk, ki predstavljajo lihe iteracije točke $x$ ležijo v množici $\mathcal{O}_L$. Zato orbita točke $x$ vsebuje $2k = l$ različnih točk in je tudi perioda točke $x$ za funkcijo $f$ enaka $l$. Sklepamo, da je zanka~\eqref{lloop} elementarna, kar zakluči dokaz.
\end{proof}

\end{document}