\documentclass[../TG_magistrsko_delo_sections.tex]{subfiles}
\graphicspath{{\subfix{../images/}}}

\begin{document}
Po zaklučenem dokazu si lahko postavimo vprašanje, kako se spremenijo posledice izreka, če spremenimo njegove predpostavke. Glede na to, kako spremenimo predpostavke, lahko pridemo do različnih posplošitev izreka. V primeru izreka Šarkovskega že obstaja več posplošitev. Nekatere obravnavajo izrek za nezvezne funkcije, ki ustrezajo določenim pogojem, druge pa preučujejo zvezne funkcije, ki so definirane na različnih topoloških prostorih. V tem primeru se lahko vprašamo, kakšna ureditev naravnih števil, če ta obstaja, opiše prisotnost periodičnih točk zvezne funkcije $f:X \to X$, ki slika nek topološki prostor $X$ nazaj vase. Natančneje, iščemo relacijo $\triangleleft_X$ z lastnostjo: če je $m$ perioda za zvezno funkcijo $f:X \to X$, potem je vsako naravno število $l$, za katero je $l \triangleleft_X m$, tudi perioda za funkcijo $f$. Če je relacija $\triangleleft_X$ enaka relaciji Šarkovskega, ki smo jo spoznali v definiciji~\ref{def:ureditev-sark}, pravimo, da je prostor $X$ \emph{prostor Šarkovskega}. To poglavje bomo namenili temu, da bomo spoznali nekaj prostorov Šarkovskega in tudi nekaj prostorov, ki to niso. S primeri in protiprimeri bomo poskušali ugotoviti katere topološke lastnosti imajo prostori Šarkovskega.
Preden začnemo s preučevanjem različnih prostorov Šarkovskega se prepričajmo, da je lastnost biti prostor Šarkovskega tudi topološka lastnost. 
\begin{trditev}
Lastnost biti prostor Šarkovskega je topološka lastnost. To pomeni, če je $X$ prostor Šarkovskega in je prostor $Y$ homeomorfen prostoru $X$, potem je tudi $Y$ prostor Šarkovskega.
\end{trditev}
\begin{proof}
Naj bo prostor $X$ prostor Šarkovskega in naj bo prostor $Y$ homeomorfen prostoru $X$. Naj bo $h : X \to Y$ homeomorfizem med prostoroma $X$ in $Y$. Funkciji $f : Y \to Y$ in $g = h^{-1} \circ f \circ h : X \to X$ imata enake periode, zato je $Y$ tudi prostor Šarkovskega.
\end{proof}

%Nove prostore, ki so prostori Šarkovskega, lahko poiščemo tudi s pomočjo nekaterih zveznih preslikav. Definirajmo tako preslikavo, ki nam pomaga konstruirati nove prostore Šarkovskega.

%\begin{definicija}
%Naj bo $X$ topološki prostor in $A \subseteq X$ njegov podprostor. Zvezni preslikavi $r : X \to A$ pravimo \emph{retrakcija}, če je preslikava $r|_A = id_A$. Podprostor $A \subseteq X$ je \emph{retrakt} prostora $X$, če obstaja retrakcija $r: X \to A$.
%\end{definicija}

%S pomočjo retrakcije lahko pridemo do novih prostorov Šarkovskega.

%\begin{trditev}
%Retrakt prostora Šarkovskega je prostor Šarkovskega.
%\end{trditev}
%\begin{proof}
%Denimo, da je prostor $X$ prostor Šarkovskega in prostor $A \subseteq X$ retrakt prostora $X$. Naj bo $r : X \to A$ retrakcija $i : X \to A$ inkluzija. Če je funkcija $f : A \to A$ zvezna, ima funkcija $g = i \circ f \circ r : X \to X$ enake periodične točke, kot funkcija $f$. Zato je tudi prostor $A$ prostor Šarkovskega.
%\end{proof}

Iz prejšnjih poglavji je razvidno, da so tipični predstavniki prostorov Šarkovskega množica realnih števil in intervali v realnih številih. Poglejmo si primer prostora, ki ni prostor Šarkovskega.



S podobnim sklepanjem kot zgoraj lahko za nekatere prostore hitro preverimo, da niso prostori Šarkovskega. To naredimo tako, da poiščemo kakšno os $n$-kratne rotacijske simetrije, kjer je $n$ naravno število večje od $2$. Pri takih primerih lahko hitro ugotovimo, da ima rotacija za kot $\varphi = \frac{360^\circ}{n}$, za $n>2$, točke periode $n$ in morda tudi točke periode $1$, nima pa točk periode $2$. Zaradi tega taki prostori ne morejo biti prostori Šarkovskega. Primeri takih prostorov so npr. sfera, krogla, torus \dots 



Na začetku primera smo enostavno pokazali, da disjunktna unija dveh intervalov ni prostor Šarkovskega. Z zelo podobno idejo lahko pokažemo, da je vsak prostor Šarkovskega povezan.

%\begin{definicija}
%Topološki prostor $X$ je \emph{nepovezan}, če obstajata neprazni odprti podmnožici $U, V \subset X$, ki zadoščata spodnjima pogojema:
%\begin{enumerate}
%\item $U \cup V = X$ in 
%\item $U \cap V \neq \emptyset$.
%\end{enumerate}
%Prostor, ki ni nepovezan je \emph{povezan}.
%\end{definicija}

\begin{trditev}
Prostor Šarkovskega je povezan.
\end{trditev}
\begin{proof}
Naj bo prostor $X$ disjunktna unija nepraznih prostorov $A$ in $B$ in naj bosta $a \in A$ in $b \in B$ poljubni točki tega prostora. Definiramo funkcijo $f:X \to X$ s predpisom:
\[ f(x) = \begin{cases}
  a, & \mbox{ če $x \in B $}\\
  b ,& \mbox{ če $x \in A$.}
  \end{cases}
  \]
Funkcija $f$ ima samo dve periodični točki. To sta točki $a$ in $b$. Obe pa imata periodo 2. Ker nobena točka prostora $X$ ni fiksna točka za funkcijo $f$, prostor $X$ ni prostor Šarkovskega.
\end{proof}

Pokazali smo, da je vsak prostor Šarkovskega povezan prostor. V nadaljevanju bomo ponovili kakšni prostori so s potmi povezani, lokalno povezani in lokalno s potmi povezani. S primeri se bomo prepričali, da obstajajo prostori Šarkovskega, ki imajo te lastnosti, in tudi prostori Šarkovskega, ki teh lastnosti nimajo.

%Najbolj tipični primeri prostorov Šarkovskega so intervali v realnih številih. Ti so povezani s potmi, lokalno povezani in lokalno povezani s potmi. Poglejmo si primer prostora Šarkovskega, ki ni 

\begin{definicija}
Topološki prostor $X$ je \emph{s potmi povezan}, če za poljubni točki $a, b \in X$ obstaja zvezna preslikava $\gamma:[0, 1] \to X$, za katero je $\gamma(0) = a$ in $\gamma(1) = b$. Preslikavi $\gamma$ rečemo tudi pot med točkama $a$ in $b$.
\end{definicija}

Primeri s potmi povezanih prostorov so intervali v realnih številih, krožnica, disk v $\R^2$\dots
\begin{trditev}
Lastnost biti s potmi povezan je strožji pogoj, kot biti povezan. To pomeni, da je vsak s potmi povezan prostor tudi povezan.
\end{trditev}

\begin{dokaz}
Naj bo $X$ s potmi povezan prostor. Dokazovali bomo s protislovjem. Denimo, da prostor $X$ ni povezan. Potem obstajata neprazni odprti množici $U, V \subseteq X$, za kateri veljata naslednji lastnosti:
\begin{enumerate}
\item $U \cup V = X$ in 
\item $U \cap V \neq \emptyset$.
\end{enumerate}
Množici $U$ in $V$ sta neprazni, zato si lahko izberemo točki $a\in U$ in $b\in V$. Prostor $X$ je s potmi povezan, kar pomeni, da obstaja pot $\gamma : [0, 1] \to X$, za katero je $\gamma(0) =a$ in $\gamma(1)=b$. Sedaj bomo obravnavali množici $\gamma^{-1}(U)$ in $\gamma^{-1}(V)$. Množici sta disjunktni podmnožici intervala $[0, 1]$, njuna unija pa je enaka interevalu $[0, 1]$. Obe množici sta odprti v $[0, 1]$, saj je pot $\gamma$ zvezna funkcija. Ker je $0$ element množice  $\gamma^{-1}(U)$ in $1$ element množice $\gamma^{-1}(V)$, tvorita ti dve množici separacijo povezane množice $[0, 1]$, kar je protislovje. Torej je prostor $X$ povezan.
\end{dokaz}

Implikacija v drugo smer ne drži. Torej, če je nek topološki prostor $X$ povezan, ne moremo sklepati, da je tudi s potmi povezan. Poglejmo si primer povezanega prostora, ki pa ni s potmi povezan.

\begin{definicija}
Naj bo $C$ grapf funkcije $\sin\left(\frac{\pi}{x}\right)$ na intervalu $x \in (0 , 1]$ in naj bo $A$ daljica $\{ 0 \} \times [-1 , 1]$. \emph{Varšavski lok} je prostor $X = C \cup A$.
\end{definicija}

Prepričajmo se, da je Varšavski lok povezan prostor. Množica $C$ je homeomorfna intervalu, zato je povezana množica in cela leži v neki komponenti za povezanost. Enako velja za množico $A$. Ker vsaka okolica točke $(0, 0) \in A$ vsebuje tudi točke iz množice $C$, ležita obe množici v isti komponenti za povezanost, zato ima prostor $X$ samo eno komponento za povezanost, kar pomeni, da je povezan.
Pokažimo, da prostor ni s potmi povezan. Poskusimo poiskati pot med točkama $(0, 0) \in A$ in $(1, 0) \in C$.

\begin{figure}[h]
  \centering
  \includegraphics{glavnik.pdf}
% \caption[caption za v kazalo]{Dolg caption pod sliko}
  \caption[Primer vektorske slike.]{Relacije pokritja v trditvi~\ref{trd:pokritja} lahko prikažemo z grafom.}
  \label{fig:varsavski_lok}
\end{figure}

\begin{definicija}
Prostor $X$ je lokalno povezan prostor, če za vsako točko $x \in X$ in vsako odptro množico $U \subseteq X$, ki vsebuje točko $x$, obstaja taka odprta povezana množica $V \subseteq X$, da je $x \in V \subseteq U$. Prostoru, ki ni lokalno povezan pravimo lokalno nepovezan prostor.
\end{definicija}

poglejmo si nekaj primerov:
\begin{primer}
Odprti disk v $\R^2$ je povezan in lokalno povezan prostor.
\end{primer}

\begin{primer}
Poglejmo si nepovezan prostor, sestavljen iz treh komponent za povezanost kot prikazuje slika. Prostor je lokalno povezan, saj za vsako točko $x \in X$ in vsako njeno odprto okolico $U\subseteq X$ obstaja taka povezana okolica $V$, da je $x \in V \subseteq U$.
\begin{figure}[h]
  \centering
  \includegraphics{nepov-lokpov.pdf}
% \caption[caption za v kazalo]{Dolg caption pod sliko}
  \caption[Primer vektorske slike.]{Relacije pokritja v trditvi~\ref{trd:pokritja} lahko prikažemo z grafom.}
  \label{fig:varsavski_lok}
\end{figure}
\end{primer}

Morda bi na hitro pomislili, da je lokalna povezanost strožji pogoj kot povezanost. To bi pomenilo, da so vsi lokalno povezani prostori tudi povezani. Prepričajmo se, da to ni res. Poglejmo si primer povezanega prostora, ki ni lokalno povezan.

\begin{figure}[h]
  \centering
  \includegraphics{glavnik.pdf}
% \caption[caption za v kazalo]{Dolg caption pod sliko}
  \caption[Primer vektorske slike.]{Relacije pokritja v trditvi~\ref{trd:pokritja} lahko prikažemo z grafom.}
  \label{fig:varsavski_lok}
\end{figure}

\begin{definicija}
Prostor $X$ je lokalno s potmi povezan prostor, če za vsako točko $ x\in X$ in vsako odprto množico $V$, ki vsebuje $x$, obstaja odprta in s potmi povezana okolica, ki vsebuje $x$ in je vsebovana v množici $V$.
\end{definicija}

\begin{primer}
Naj bo $C$ grapf funkcije $\sin\left(\frac{\pi}{x}\right)$ na intervalu $x \in (0 , 1]$ in naj bo $A$ daljica $\{ 0 \} \times [-1 , 1]$. \emph{Varšavski lok} je prostor $X = C \cup A$.
Prostor $X$ je povezan, kompakten in ima dve komponenti za povezavost s potmi. Komponenta $A$ je homeomorfna zaprtemu intervalu, komponenta $C$ pa je homeomorfna polodprtemu intervalu. Prostor je prikazan na sliki~\ref{fig:varsavski_lok}.
Prepričajmo se, da je prostor $X$ povezan. Ker je množica $C$ povezana, celotna leži v neki komponenti za povezanost $V$ prostora $X$. Poglejmo točko iz $x = (0, 0) \in  A$. Naj bo $\delta$ poljubno majhno pozitivno število in množica $U$ delta okolica točke $x$ v prostoru $X$. Za naravno število $k > \frac{1}{\pi \delta}$ točka $(\frac{1}{k \pi}, 0)$ leži v množici $C$ in v množici $U$, kar pomeni, da točka $(0, 0)$ leži v komponenti za povezanost $V$. Ker je množica $A$ povezana množica, cela leži v množici $V$, zato ima prostor $X$ eno samo komponento za povezanost in je povezan prostor.
\end{primer}

\begin{trditev}
Varšavski lok je prostor Šarkovskega.
\end{trditev}
\begin{proof}
Varšavski lok zapišimo kot $X = C \cup A$, kjer je C krivulja $\{(x, \sin(\frac{\pi}{x})), x\in [0, 1]\}$ in $A= \{0\} \times [-1, 1]$. Naj bo $x \in X$ točka s periodo $n$ in naj bo $m$ tako naravno število, da velja relacija $m \triangleleft n$. Pokazali bomo, da obstaja točka $y$ s periodo $m$. Ker je funkcija $f$ zvezna, se ne more zgoditi, da ja $f(A) \subseteq C$ in $f(C) \subseteq A$. V tem primeru bi bila množica $f(C)$ kompaktna množica, ki nima skupne točke z množico $C$. Ker ima prostor $\R$ lastnost $T_4$, obstajata disjunktni odprti množici, $U$ in $V$, za kateri velja $f(C) \subseteq U$, $f(A) \subseteq C \subseteq V$, kar predstavlja separacijo prostora $f(X)$, kar pa ni mogoče, saj je prostor $X$ povezan. Množica $f(X)$ je slika povezanega prostora z zvezno funkcijo in je tudi povezana. Če je $f(A) \subseteq C$, potem je zaradi zveznosti funkcije $f$ tudi $f(C) \subseteq C$. Ker je $X$ kompaktna množica in je slika kompaktne množice z zvezno preslikavo kompaktna, je množica $f(X)$ kompaktna povezana podmnožica množice $C$. Množica $C$ je homeomorfna intervalu, zato je tudi $f(X)$ homeromorfna intervalu, kar pomeni, da je tudi $f(X)$ prostor Šarkovskega. Zato obstaja točka $y$ s periodo $m$. Če je $f (C) \subseteq A $, potem je tudi $f (A) \subseteq A $ in vsaka periodična točka funkcije $f$ leži v $A$.  Zopet je množica $f(X)$ povezana kompaktna podmnožica homeomorfna intervalu. Torej obstaja točka $y$ s periodo $m$. 
Dokazati moramo še primer, ko je $f (A) \subseteq A $ in $f (C) \subseteq C $. Ker sta prostora $A$ in $C$ homeomorfna intervalu, sta prostora Šarkovskega, kar pomeni, da zagotovo obstaja točka $y \in X$, ki leži v isti komponenti za povezanost s potmi kot točka $x$, s periodo $m$.
\end{proof}

\begin{figure}[h]
  \centering
  \includegraphics{varsavskilok.pdf}
% \caption[caption za v kazalo]{Dolg caption pod sliko}
  \caption[Primer vektorske slike.]{Relacije pokritja v trditvi~\ref{trd:pokritja} lahko prikažemo z grafom.}
  \label{fig:varsavski_lok}
\end{figure}

\begin{figure}[h]
  \centering
  \includegraphics{varsavska_kroznica.pdf}
  \caption[Varšavska krožnica]{Slika prikazuje primer varšavske krožnice določene s predpisom $p$ iz definicije.}
  \label{fig:varšavski}
\end{figure}


\begin{definicija}\label{def:vk}
varšavska krožnica je topološki prostor, ki ga lahko definiramo na naslednji način:
$$S_W = \left\{\left(x, \sin\left(\frac{\pi}{x}\right)\right); 0 < x \leq 1\right\} \cup \{(0, y); -1 \leq x \leq 1\} \cup C,$$
kjer je C zvezna krivulja, ki povezuje točki $(0,1)$ in $(0,0)$ in ne seka preostalega dela Waršavske krožnice.
Množici, ki jo lahko parametriziramo na naslednji način:
\[ p(t) = \begin{cases}
  (t, \sin(\frac{\pi}{t}), & \mbox{ če $t \in (0, 1) $}\\
 (\cos(\frac{3\pi x}{2}-\frac{\pi}{2})+1, \sin(\frac{3\pi x}{2}-\frac{\pi}{2})-1). & \mbox{ če $t \in [1, 2]$}\\
  (0, 2t-5), & \mbox{ če $t \in (2, 3]$.}
  \end{cases}
  \]
\end{definicija}

\begin{trditev}
Varšavska krožnica je prostor Šarkovskega.
\end{trditev}

\begin{tikzcd}
A \arrow{d} \arrow{r}[near start]{\phi}[near end]{\psi}
& B \arrow[red]{d}{\xi} \\
C \arrow[red]{r}[blue]{\eta}
& D
\end{tikzcd}

\begin{proof}
Varšavsko krožnico $X$ lahko parametriziramo z zvezno bijektivno preslikavo $p:I \to X$. Naj bo $f: X \to X$ zvezna funkcija. Ker je funkcija $p$ bijektivna, je funkcija $\widehat{f} = p^{-1} \circ f \circ p : I \to I$ dobro definirana. Trdimo, da je funkcija $\widehat{f}$ zvezna. Ker je funkcija $p$ bijekcija, imata funkciji $f$ in $\widehat{f}$ enake periode. Ker je interval $I$ prostor Šarkovskega, je tudi $X$ prostor Šarkovskega. 

Prepričati se moramo samo še, da je funkcija $\widehat{f}$ res zvezna. Naj bo $t \in I$ poljubna točka intervala $I$ in naj bo $U \in I$ odprta krogla okoli točke $\widehat{f}(t) = (p^{-1} \circ f \circ p)(t)$. Množoco robnih točk krogle $U$ označimo z $A$. Velja $|A| \leq 2$. Ker ima $X$ lastnost $T_2$, so točke zaprte množice in zato je množica $X - p(A)$ odprta podmnožica prostora $X$, ki vsebuje točko $(f \circ p)(t)$. Povezano komponento množice $(f \circ p)^{-1}$, ki vsebuje točko $t$ označimo z $W$. Množica $W$ je odprta podmnožica intervala $I$, saj je funkcija $(f \circ p)$ zvezna. Sedaj obravnavamo množico $\left(p \circ \widehat{f}\right) (W) = (f \circ p)(W) \subseteq X - p(A)$. Ker je množica $W$ povezava s potmi, je vsebovana v tisti komponenti za povezanost s potmi množice $X-p(A)$, ki vsebuje $(f \circ p)(t)$. 
Prepričali se bomo, da je ta komponenta kar enaka $p(U)$. To je res, saj komponenta vsebuje $p(U)$ in ne more vsebovati nobene druge točke. Denimo, da vsebuje še kakšno drugo točko $x$. Potem vsebuje pot od $x$ do $p(t)$. Ta pot pa  
\end{proof}

\begin{figure}[h]
  \centering
  \includegraphics{vk-shark.pdf}
  \caption[Varšavska krožnica]{Slika prikazuje primer varšavske krožnice določene s predpisom p iz definicije.}
  \label{fig:varšavski}
\end{figure}

\end{document}